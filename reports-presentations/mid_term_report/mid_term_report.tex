% define document class and set class options
\documentclass[11pt, a4paper]{article}

%===============================================
% *** LIST OF GENERAL PACKAGES *** %
% language package (french option available as well)
\usepackage[english]{babel}

% input fonts
\usepackage[utf8]{inputenc}
\usepackage[T1]{fontenc}
\usepackage{csquotes}

% customize default page geometry (size of margins, ...)
\usepackage{geometry}
\geometry{hmargin=2cm,vmargin=3cm}
\usepackage{titling}

\usepackage{titlesec}

% writing style
\usepackage{lmodern}
\usepackage{times}	 

% include images (image extensions & paths)
\usepackage{graphicx}
\DeclareGraphicsExtensions{.pdf,.jpg,.png,.eps}
\graphicspath{{figs/}}

\usepackage{wrapfig}

% display hyperlinks in the text
\usepackage{hyperref}  %
\hypersetup{colorlinks,%
            citecolor=red,%
            filecolor=black,%
            linkcolor=blue,%
            urlcolor=blue,%
            breaklinks=true}
            
% customize enumerated lists
\usepackage{enumitem}

% lipsum command to fill in the document (for visualisation purposes)
\usepackage{lipsum}

%===============================================
% *** CAPTIONS AND SUBFIGURES *** %
\usepackage{subcaption}

%===============================================
% *** TABLES *** %
\usepackage{booktabs}
\usepackage{multirow}

%=============================================== 
% *** BIBLIOGRAPHY (FOR BIBLATEX) *** %
\usepackage[
    backend=biber,
    style=ieee,
    ]{biblatex}

\usepackage{bookmark}
\addbibresource{bgtg_lib.bib}

% citation style: apa, ieee, authoryear, ...

%===============================================
% *** MATHS / PHYSICS PACKAGES *** %
\usepackage{amsfonts,amssymb,amsmath,amsthm}
\usepackage{mathrsfs}
\usepackage{mathtools}  % for DeclarePairedDelimiter
\usepackage{siunitx}    % notation physical units

% notations (shortcuts)
\newcommand{\bs}[1]{\boldsymbol{#1}}
\newcommand{\nbpix}{N}
\DeclareMathOperator{\prox}{prox}
\DeclareMathOperator{\sgn}{sgn}
\DeclareMathOperator*{\argmin}{argmin}
\DeclarePairedDelimiter{\norm}{\lVert}{\rVert}

%===============================================
% *** TITLE, AUTHOR AND DATE *** %
% \pretitle{\begin{center}\fontsize{30bp}{30bp}\selectfont}
% \posttitle{\par\end{center}}

% \preauthor{\begin{center}\fontsize{14bp}{14bp}\selectfont}
% \postauthor{\par\end{center}}
%===============================================
% *** MAIN DOCUMENT *** %
\begin{document}

\begin{titlepage}
\centering
% Logos
\begin{minipage}{.25\linewidth}
\includegraphics[width=\linewidth]{logos/LOGO_SU_HORIZ_SEUL.jpg}
\end{minipage}
\hfill
\begin{minipage}{.25\linewidth}
\centering
\includegraphics[width=0.5\linewidth]{logos/Logo_Télécom_Paris.jpg}
\end{minipage}
\hfill
\begin{minipage}{.25\linewidth}
\includegraphics[width=\linewidth]{logos/IRCAM.CP.jpg}
\end{minipage}

\vfill

% Subtitle

{\large Internship Report \\ Written in August 2022 \vspace{1\baselineskip}}

% Title
\begin{minipage}{\linewidth}
\huge
\bfseries
\centering
\rule{\linewidth}{1.5pt}\\
Automatic effect recognition and configuration for timbre reproduction\\[-3mm]
\rule{\linewidth}{1.5pt}
\end{minipage}

\vfill

% Persons involved 
\begin{minipage}{.45\linewidth}
\textit{Intern:}\\
Alexandre \textsc{D'Hooge}\\
\texttt{alexandre.dhooge@atiam.fr}
\end{minipage}
\hfill
\begin{minipage}{.45\linewidth}
\flushright
\textit{Supervisor:}\\
Gaëtan \textsc{Hadjeres}\\
\texttt{gaetan.hadjeres@sony.com}
\end{minipage}

\vfill

% Period
March 1st --- August 31st
\vspace{1\baselineskip}\\
\begin{flushleft}
\textbf{Key-words ---} audio effects, music information retrieval, differentiable signal processing, sound matching, computer music;
\vspace{.5\baselineskip}\\
\textbf{Mots-clés \hspace{1ex}---} effets audio, traitement de l'information musicale, traitement du signal différentiable, reproduction sonore, musique assistée par ordinateur.
\end{flushleft}
\vfill

% More logos
\begin{minipage}{.25\linewidth}
\includegraphics[height=3.5cm]{logos/ATIAM_logo_ROND_RVB.jpg}
\end{minipage}
\hspace{5cm}
\begin{minipage}{.25\linewidth}
\includegraphics[height=5cm]{logos/logoCSL512.png}
\end{minipage}

\end{titlepage}


% \begin{figure}[t]% We use titling to put a figure on top of the title page
%     \centering
%     \includegraphics[width=.5\linewidth]{logos_empile}
% \end{figure}

% \title{Bass guitar tablature conditional generation for guitar accompaniment in western popular music}

% \author{Olivier \textsc{Anoufa} \\  University of Lille, France \\ Master 2 Data Science: Research project}

% {\let\newpage\relax\maketitle}

\newpage

\section*{Introduction}

% Context and motivation
Natural language processing methods for the generation of symbolic music is a field of research that has seen great development in the last years.
The transformer architecture, introduced by Vaswani and al. in 2017, has been used to generate scores for various instruments, in diverse styles and genres\cite{vaswaniAttentionAllYou2023, leNaturalLanguageProcessing2024}.
An under publication user study led by Bacot and al. showed a potential need for accompaniment generation tools for guitarists.
The questionnaire was answered by 31 guitarists composers, and 7 of them followed up with an interview.
During the interviews, the focus was made on bass guitar lines and drum parts generation without necessarily having to be familiar with the instrument.
Indeed, guitarists generally tend to write basic bass chords to accompany them, and an AI tool could perform this functional task for them\cite{bacot_tablature_2025}.


% Objectives
This need is the starting point of the project that I chose, proposed by the Algomus team which is part of the CRIStAL laboratory.
In this project, we focus on the conditional aspect of symbolic music generation, for an instrument that has not been thoroughly studied yet: the bass guitar.
More precisely, we aim at generating bass guitar tablatures given other instruments' scores, in the context of western popular music.
Our goal will be to try several combinations of instruments and to evaluate the quality of the generated tablatures both numerically and using the help of musicians.


% Terms definition (tablature, add the period when it was used)
% Difference between guitar and bass (role in the music)
To understand better what is at stake in this challenge, we will first define precisely the role of bass guitar in the context of western popular music.
As human ear has a better perception of pulses in low frequencies,
it is said that the bass guitar is part of the rythmic section of the band (together with the drums)\cite{hoveSuperiorTimePerception2014}.
However, bass guitar also performs a harmonic - and sometimes melodic - role in the music, sustaining the lead instruments and adding depth to the music.
% add a fig Time is running out, melodic and rythmic bass
\begin{figure}[h!]
    \centering
    \begin{minipage}{0.45\textwidth}
        \centering
        \includegraphics[width=.5\linewidth]{figs/rythmic_tab_TIRO.png}
        \caption{Rythmic extract}
    \end{minipage}%
    \hfill
    \begin{minipage}{0.45\textwidth}
        \centering
        \includegraphics[width=.5\linewidth]{figs/melodic_tab_TIRO.png}
        \caption{Melodic extract}
    \end{minipage}
    \label{fig:bass_tab_TIRO}
    \caption{Rythmic and melodic extracts in Time is Running Out by Muse}
\end{figure}

\ref{fig:bass_tab_TIRO} shows a tablature and score extract of both a rythmic and a more melodic bass.
Scores display the notes to play, while tablatures show the fingering to use on the instrument.
More precisely, each line of the tablature represent a string of the instrument, and the numbers represent the fret to press on the string.
For instance in the figure on the right, the bassist will start by playing the the fourth string with the first fret.
Tablatures don't contain the rythmic information which is why they are generally combined with scores.
Else, the musician has to know the song to play it correctly.


% Abstract challenges: (high-level such as: propose an informatic representation of music...)
Generating bass guitar tablatures presents several challenges.
At a high level, we first have to scrap and preprocess large amounts of data in the form of music scores.
Then we have to come up with an informatic representation of music that is adapted to the task of generating bass guitar tablatures.
That is, a way to encode music scores in a meaningful way to the transformer architecture we will use.
We will start by trying state of the art models but we will have to adapt and tune them to the task at hand.


\section*{State of the art}

% Challenges (tokenization, conditional generation, attention, data cleaning), training of the model)
Adapting the transformer architecture - originally applied on text - to symbolic music presents many challenges.
Especially, tokenization has to be adapted, data is much less available, and attention has to be tailored to the context\cite{leNaturalLanguageProcessing2024}.

Tokenization in the context of deep learning music generation has been discussed by several previous works
\cite{agarwalStructureinformedPositionalEncoding2024, makrisConditionalDrumsGeneration2022, sarmentoDadaGPDatasetTokenized2021, hsiaoCompoundWordTransformer2021, cournutEncodagesTablaturesPour2020}


% State of the art (survey, DadaGP, Compound Gen, Structure-informed positional encoding, gen coherent drum accompaniment)
% ResearchRabbit to find potential other articles

% What we have done (data preprocessing, getting the rythmic instrument, retrieving the tokens, baseline using dadagp (GuitarCTRL))
% draw io fig of the model


% Workplan (what we are going to do)

\newpage

%\nocite{*} % force display of the full content of the .bib file, w/o any citation in the document
\printbibliography% references: print bibliography (with bibtex file)

\end{document}