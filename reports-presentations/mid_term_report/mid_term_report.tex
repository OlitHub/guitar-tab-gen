% define document class and set class options
\documentclass[11pt, a4paper]{article}

%===============================================
% *** LIST OF GENERAL PACKAGES *** %
% language package (french option available as well)
\usepackage[english]{babel}

% input fonts
\usepackage[utf8]{inputenc}
\usepackage[T1]{fontenc}
\usepackage{csquotes}

% customize default page geometry (size of margins, ...)
\usepackage{geometry}
\geometry{hmargin=2cm,vmargin=3cm}
\usepackage{titling}

\usepackage{titlesec}

% writing style
\usepackage{lmodern}
\usepackage{times}	 

% include images (image extensions & paths)
\usepackage{graphicx}
\DeclareGraphicsExtensions{.pdf,.jpg,.png,.eps}
\graphicspath{{figs/}}

\usepackage{wrapfig}

% display hyperlinks in the text
\usepackage{hyperref}  %
\hypersetup{colorlinks,%
            citecolor=red,%
            filecolor=black,%
            linkcolor=blue,%
            urlcolor=blue,%
            breaklinks=true}
            
% customize enumerated lists
\usepackage{enumitem}

% lipsum command to fill in the document (for visualisation purposes)
\usepackage{lipsum}

%===============================================
% *** CAPTIONS AND SUBFIGURES *** %
\usepackage{subcaption}

%===============================================
% *** TABLES *** %
\usepackage{booktabs}
\usepackage{multirow}

%=============================================== 
% *** BIBLIOGRAPHY (FOR BIBLATEX) *** %
\usepackage[
    backend=biber,
    style=ieee,
    ]{biblatex}

\usepackage{bookmark}
\addbibresource{bgtg_lib.bib}

% citation style: apa, ieee, authoryear, ...

%===============================================
% *** MATHS / PHYSICS PACKAGES *** %
\usepackage{amsfonts,amssymb,amsmath,amsthm}
\usepackage{mathrsfs}
\usepackage{mathtools}  % for DeclarePairedDelimiter
\usepackage{siunitx}    % notation physical units

% notations (shortcuts)
\newcommand{\bs}[1]{\boldsymbol{#1}}
\newcommand{\nbpix}{N}
\DeclareMathOperator{\prox}{prox}
\DeclareMathOperator{\sgn}{sgn}
\DeclareMathOperator*{\argmin}{argmin}
\DeclarePairedDelimiter{\norm}{\lVert}{\rVert}

%===============================================
% *** TITLE, AUTHOR AND DATE *** %
% \pretitle{\begin{center}\fontsize{30bp}{30bp}\selectfont}
% \posttitle{\par\end{center}}

% \preauthor{\begin{center}\fontsize{14bp}{14bp}\selectfont}
% \postauthor{\par\end{center}}
%===============================================
% *** MAIN DOCUMENT *** %
\begin{document}

\begin{titlepage}
\centering
% Logos
\begin{minipage}{.25\linewidth}
\includegraphics[width=\linewidth]{logos/LOGO_SU_HORIZ_SEUL.jpg}
\end{minipage}
\hfill
\begin{minipage}{.25\linewidth}
\centering
\includegraphics[width=0.5\linewidth]{logos/Logo_Télécom_Paris.jpg}
\end{minipage}
\hfill
\begin{minipage}{.25\linewidth}
\includegraphics[width=\linewidth]{logos/IRCAM.CP.jpg}
\end{minipage}

\vfill

% Subtitle

{\large Internship Report \\ Written in August 2022 \vspace{1\baselineskip}}

% Title
\begin{minipage}{\linewidth}
\huge
\bfseries
\centering
\rule{\linewidth}{1.5pt}\\
Automatic effect recognition and configuration for timbre reproduction\\[-3mm]
\rule{\linewidth}{1.5pt}
\end{minipage}

\vfill

% Persons involved 
\begin{minipage}{.45\linewidth}
\textit{Intern:}\\
Alexandre \textsc{D'Hooge}\\
\texttt{alexandre.dhooge@atiam.fr}
\end{minipage}
\hfill
\begin{minipage}{.45\linewidth}
\flushright
\textit{Supervisor:}\\
Gaëtan \textsc{Hadjeres}\\
\texttt{gaetan.hadjeres@sony.com}
\end{minipage}

\vfill

% Period
March 1st --- August 31st
\vspace{1\baselineskip}\\
\begin{flushleft}
\textbf{Key-words ---} audio effects, music information retrieval, differentiable signal processing, sound matching, computer music;
\vspace{.5\baselineskip}\\
\textbf{Mots-clés \hspace{1ex}---} effets audio, traitement de l'information musicale, traitement du signal différentiable, reproduction sonore, musique assistée par ordinateur.
\end{flushleft}
\vfill

% More logos
\begin{minipage}{.25\linewidth}
\includegraphics[height=3.5cm]{logos/ATIAM_logo_ROND_RVB.jpg}
\end{minipage}
\hspace{5cm}
\begin{minipage}{.25\linewidth}
\includegraphics[height=5cm]{logos/logoCSL512.png}
\end{minipage}

\end{titlepage}


% \begin{figure}[t]% We use titling to put a figure on top of the title page
%     \centering
%     \includegraphics[width=.5\linewidth]{logos_empile}
% \end{figure}

% \title{Bass guitar tablature conditional generation for guitar accompaniment in western popular music}

% \author{Olivier \textsc{Anoufa} \\  University of Lille, France \\ Master 2 Data Science: Research project}

% {\let\newpage\relax\maketitle}

\newpage

\section*{Introduction}

Natural language processing methods for the generation of symbolic music is a field of research that has seen great development in the last years.
The transformer architecture, introduced by Vaswani and al. in 2017, has been used to generate scores for various instruments, in diverse styles and genres\cite{vaswaniAttentionAllYou2023, leNaturalLanguageProcessing2024}.
However, adapting the transformer architecture - originally applied on text - to symbolic music presents many challenges.
Tokenization has to be adapted, data is much less available, and attention has to be tailored to the context\cite{leNaturalLanguageProcessing2024}.

% Talk about the team
% context about the type of music used


% Interviews with guitarists led to this need: guitarists generally make basic bass chords to accompany them. (Article Baptiste)
% This tool would help them build an accompaniment


% Abstract challenges: (high-level such as: propose an informatic representation of music...)
% Objectives
In this project, we focus on the conditional aspect of symbolic music generation, for an instrument that has not been thoroughly studied yet: the bass guitar.
More precisely, we aim to generate bass guitar tablatures given other instruments' scores. Our goal will be to try several combinations of instruments and to evaluate the quality of the generated tablatures.

% Terms definition (tablature, add the period when it was used)
Tablatures are a way to represent music for string instruments. They contain information about the fingering to use to play the notes and the rhythm.
% add a fig (Kryptonite - 3 Doors down - electric guitar introduction)

% Difference between guitar and bass (role in the music)


% State of the art (survey, DadaGP, Compound Gen, Structure-informed positional encoding, gen coherent drum accompaniment)
% ResearchRabbit to find potential other articles

% Challenges (tokenization, conditional generation, attention, data cleaning), training of the model)

% What we have done (data preprocessing, getting the rythmic instrument, retrieving the tokens, baseline using dadagp (GuitarCTRL))
% draw io fig of the model


% Workplan (what we are going to do)

\newpage

%\nocite{*} % force display of the full content of the .bib file, w/o any citation in the document
\printbibliography% references: print bibliography (with bibtex file)

\end{document}