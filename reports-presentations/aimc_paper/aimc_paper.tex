% License: LaTeX Project Public License 1.3c
% file aimc2025.tex 
% This is the LaTeX source for the instructions to authors using
% the LaTeX2e package 'aimc22.sty' for contributions to
% the Conference on AI Music Creativity.

\documentclass{article}

% if you need to pass options to natbib, use, e.g.:
%     \PassOptionsToPackage{numbers, compress}{natbib}
% before loading aimc2025

% load package
\usepackage{aimc2025}

\usepackage[utf8]{inputenc} % allow utf-8 input
\usepackage[T1]{fontenc}    % use 8-bit T1 fonts
\usepackage{hyperref}       % hyperlinks
\hypersetup{
  colorlinks   = true, %Colours links instead of ugly boxes
  urlcolor     = blue, %Colour for external hyperlinks
  linkcolor    = black, %Colour of internal links
  citecolor   = black %Colour of citations
}
\usepackage{url}            % simple URL typesetting
\usepackage{booktabs}       % professional-quality tables
\usepackage{amsfonts}       % blackboard math symbols
\usepackage{nicefrac}       % compact symbols for 1/2, etc.
\usepackage{microtype}      % microtypography
\usepackage{graphicx}       % figures

\title{\LaTeX{} Conditional Generation of Bass Guitar Tablature for Guitar Accompaniment in Western Popular Music}

% The \author macro works with any number of authors. There are two commands
% used to separate the names and addresses of multiple authors: \And and \AND.
%
% Using \And between authors leaves it to LaTeX to determine where to break the
% lines. Using \AND forces a line break at that point. So, if LaTeX puts 3 of 4
% authors names on the first line, and the last on the second line, try using
% \AND instead of \And before the third author name.

\author{%
  Olivier Anoufa\thanks{Use footnote for providing further information
    about author (webpage, alternative address)---\emph{not} for acknowledging
    funding agencies.} \\
  Department of Music\\
  Ecole Centrale de Lille\\
  Pittsburgh, PA 15213 \\
  \texttt{olivier.anoufa@centrale.centralelille.fr} \\
  % examples of more authors
  \And
  Coauthor \\
  Affiliation \\
  Address \\
  \texttt{email} \\
  \AND
  Coauthor \\
  Affiliation \\
  Address \\
  \texttt{email} \\
  % \And
  % Coauthor \\
  % Affiliation \\
  % Address \\
  % \texttt{email} \\
  % \And
  % Coauthor \\
  % Affiliation \\
  % Address \\
  % \texttt{email} \\
}

\begin{document}

\maketitle

\begin{abstract}
  The field of symbolic music generation has seen great advancements with the rise of transformer-based architectures.
  Addressing a specific need identified through a user study conducted on guitar players, we focus on developing AI tools to generate bass guitar tablatures conditioned on scores of other instruments in Western popular music.
  The bass guitar, a crucial component of the rhythmic and harmonic sections in music, presents a unique challenge due to its dual role in providing structure and groove.
  Building upon the tablature notation, the most popular way to share music among guitarists, this work adopts modern encoding schemes to integrate tablature representation into transformer-based models.
  To this end, the project involves preprocessing a large dataset of music scores and fine-tuning state-of-the-art transformer architectures.
  Due to the project being the first of its kind and its creative nature, the evaluation of the generated tablatures is mainly done qualitatively.
\end{abstract}

\newpage

\section{Introduction}

% Context and motivation
Natural language processing methods for the generation of symbolic music have experienced significant advancements in recent years.
The transformer architecture, introduced by Vaswani et al. in 2017, has been used to generate scores for various instruments, in diverse styles and genres\cite{vaswani_attention_2023, le_natural_2024}.
An unpublished user study conducted by Bacot et al. showed a potential need for accompaniment generation tools for guitarists.
The questionnaire was answered by 31 guitarist-composers, and 7 of them followed up with an interview.
During the interviews, several guitarists answered that they would like to be able to generate bass guitar lines and drum parts without requiring familiarity with the instruments.
Indeed, guitarists often resort to writing basic bass lines to accompany their compositions, and an AI tool could perform this functional task for them\cite{bacot_tablature_2025}.

% Objectives
This need is the starting point for this project, proposed by the Algomus team, part of the CRIStAL laboratory.
We focus on the conditional aspect of symbolic music generation, for an instrument that has not been thoroughly studied yet: the bass guitar.
Specifically, the goal is to generate bass guitar tablatures given other instruments' scores, in the context of western popular music.
Our objective is to try several combinations of conditioning instruments and to evaluate the quality of the generated tablatures both numerically and with the help of musicians.


% Terms definition (tablature, add the period when it was used)
% Difference between guitar and bass (role in the music)
To better understand what is at stake in this challenge, we will begin by precisely defining the role of the bass guitar in the context of western popular music.
Since the human ear perceives low-frequency pulses more distinctly, the bass guitar is considered part of the rhythmic section of the band (together with the drums)\cite{hove_superior_2014}.
However, the bass guitar also performs a harmonic — and sometimes melodic — role in the music, sustaining the lead instruments and adding depth to the composition.
Its adaptability makes the bass guitar a crucial component across various genres of western popular music.


Historically, tablatures have been used since the Renaissance period as a simplified notation system for string instruments like the lute.
Originating around the 16th century, tablatures were an intuitive alternative to staff notation, allowing musicians to bypass the complexity of interpreting pitches.
This system explicitly linked symbols to physical actions on the instrument, such as pressing specific frets or strings.
As a result, tablatures gained popularity among amateur and professional musicians alike.
In modern times, tools like GuitarPro have digitalized tablatures, further extending their use for learning, practice, and composition in genres ranging from classical to popular music\cite{sarmento_dadagp_2021}.

% add a fig Time is running out, melodic and rhythmic bass
\begin{figure}[h!]
    \centering
    \begin{minipage}{0.45\textwidth}
        \centering
        \includegraphics[width=.5\linewidth]{../images-figures/rhythmic_tab_TIRO.png}
    \end{minipage}%
    \hfill
    \begin{minipage}{0.45\textwidth}
        \centering
        \includegraphics[width=.5\linewidth]{../images-figures/melodic_tab_TIRO.png}
    \end{minipage}
    \caption{Rhythmic (left) and melodic (right) extracts in Time is Running Out by Muse}
    \label{fig:bass_tab_TIRO}
\end{figure}


Figure~\ref{fig:bass_tab_TIRO} shows a tablature and score extract of both a rhythmic and a more melodic bass.
Scores display the notes to play, while tablatures show the fingering to use on the instrument.
More precisely, each line of the tablature represents a string of the instrument, and the numbers indicate the fret to press on the string.
For instance, in the figure on the right, the bassist will start by playing the fourth string with the first fret pressed.
Tablatures do not contain rhythmic information, which is why they are generally combined with scores.
Otherwise, the musician must know the song to play it correctly.


% Abstract challenges: (high-level such as: propose an informatic representation of music...)
Generating bass guitar tablatures presents several challenges.
At a high level, we first need to scrap and preprocess large datasets of music scores.
Then, we need to design a computational representation of music that is adapted to the task of generating bass guitar tablatures.
That is, a way to encode music scores in a meaningful way for the transformer architecture we will use.
Concerning the generation, we will start by leveraging state-of-the-art models but will adapt and tune them to the task at hand.

\section{State of the art}

% Challenges (tokenization, conditional generation, attention, data cleaning), training of the model)
Adapting the transformer architecture — originally developed for text processing — to symbolic music presents many challenges.
Symbolic music datasets, unlike text corpus, are limited in both size and diversity, posing challenges for training robust models\cite{le_natural_2024}.
Tokenization must be tailored to represent pitch, duration, and dynamics, while attention mechanisms require adaptation to capture the hierarchical and temporal structure of music.
Finally, data cleaning and preprocessing steps are critical to standardize music scores and ensure the compatibility with sequence models.

% State of the art (survey, DadaGP, Compound Gen, Structure-informed positional encoding, gen coherent drum accompaniment)
% ResearchRabbit to find potential other articles

\subsection{Data availability}


Symbolic music datasets are the first necessary resource for training deep learning models, yet their availability and quality significantly vary across domains.
In music composition and generation, datasets are often limited in diversity, especially when compared to text or image datasets.
For bass guitar tablatures, the challenge is even more pronounced due to the niche nature of the instrument and the focus on other instruments in existing datasets.
% EXAMPLE OF PIANO, POP MUSIC, CLASSICAL music datasets 

The MIDI standard was predominant among symbolic music datasets for decades, allowing the development of resources like the Lakh MIDI dataset and MAESTRO dataset.
However, while these datasets offer general-purpose symbolic music, they lack the specificity required for tasks involving tablatures or instrument-specific representation.
The GuitarSet dataset, for example, focuses on acoustic guitar transcription but does not provide sufficient symbolic information for bass guitar.
Similarly, the DadaGP dataset addresses the need for multi-instrument symbolic music data in tablature format, but its emphasis is on rock and metal genres, which limits the diversity of bass guitar styles\cite{sarmento_dadagp_2021}.


Bass guitar data, especially in tablature format, suffers from a lack of standardization and availability.
Tablature files are often stored in private formats like GuitarPro or as non-standardized text files, making it challenging to preprocess them for machine learning tasks\cite{sarmento_dadagp_2021}.
Moreover, rhythm and dynamics, critical elements in bass guitar playing, are frequently absent in publicly available tablatures, complicating their utility in generative tasks.


Efforts like DadaGP illustrate the potential of GuitarPro files repositoriess to create symbolic datasets that include bass guitar parts.
Moreover DadaGP also provides a tokenized format inspired by MIDI encodings, offering a foundation for training sequence models.
On the other hand, pre-trained models on broader datasets like MAESTRO or Lakh MIDI can be fine-tuned on smaller datasets, reducing the dependency on large volumes of task-specific data\cite{makris_conditional_2022, sarmento_dadagp_2021}.
For our project, we use the DadaGP dataset. Its very specific tokenization will be discussed in the next section.


\subsection{Tokenization}

Tokenization in the context of deep learning music generation has been discussed by several previous works\cite{agarwal_structure-informed_2024, makris_conditional_2022, sarmento_dadagp_2021, hsiao_compound_2021, cournut_encodages_2020}.

% General overview of the possible startegies (NLP survey)
Tokenization in music involves converting complex musical content into a sequence of elements that can be processed computationally.
% START WITH TEXT TOKENIZATION: ON TEXT IT LOOKS LIKE THIS --> ON MUSIC IT LOOKS LIKE THAT
Much like tokenization in natural language processing (NLP), it breaks down music into manageable units, though in music, the focus is on musical features like pitch, duration, and velocity.
In symbolic music information retrieval (MIR), tokenization strategies are generally classified into two categories: time-slice-based and event-based. 
Time-slice-based tokenization divides music into fixed-time intervals, such as 16th notes, which can be represented in formats like piano rolls or multi-hot vectors, capturing simultaneous notes at each time slice.
Event-based tokenization, on the other hand, focuses on specific musical events, such as a note being played or a measure starting, often using formats like MIDI, which encode music as a series of events.
This approach can involve elementary tokens, which represent individual features like pitch or duration, or composite tokens that aggregate multiple features into one token, providing a more compact representation.
Notably, tokenization strategies like REMI (Revamped MIDI-derived events) and MIDI-like tokenization allow for consistent representation of rhythmic and pitch elements while addressing complexities like polyphony and multiple tracks in multi-instrument music\cite{le_natural_2024, cournut_encodages_2020}.


% DadaGP tokenization
The DadaGP tokenization format adopts an event-based approach, similar to other music generation models, by representing musical events as discrete tokens.
It uses a Python script with PyGuitarPro to convert GuitarPro files into a tokenized sequence, beginning with metadata tokens such as artist, downtune, tempo, and start.
Pitched instrument notes are encoded with a combination of tokens representing instrument, note, string, and fret, while rests are denoted with a separate token structure.
For drumset sounds, a specific tokenization scheme using MIDI percussion maps is employed, where each drum sound is represented by a unique token (e.g., \texttt{drums:note:36} for a kick drum).
The system also uses wait tokens to separate notes and rests, enabling the model to infer note durations without the need for explicit note-off tokens.
This approach ensures that new notes automatically silence previous ones, except in cases where ghost notes or let-ring effects are involved.
Additionally, the tokenization format records changes in musical structure, such as new measures, tempo shifts, and note effects, with each change being represented as a distinct token.
However, the tokenization does not encode dynamics or articulations. These are supposed to be inferred by the user after generation.


\begin{figure}[h!]
    \centering
    \includegraphics[width=.5\linewidth]{../images-figures/dadagp_tokenization_example_measure.png}
    \caption{Example measure with a distorted guitar, bass and drums in GuitarPro’s graphical user interface (left), and its conversion into token format (right). Figure taken from \cite{sarmento_dadagp_2021}}
    \label{fig:dadagp_tokenization}
\end{figure}


Figure \ref{fig:dadagp_tokenization} illustrates the tokenization process for a measure in a GuitarPro file, showing the conversion of a musical event into a tokenized sequence.
We notice the three first tokens are the metadata tokens. "start" token marks the beginning of the song.
Afterwards, each measure is announced by a "new\_measure" token.
In between, the instruments' notes are encoded using the tablature notation for the guitars.
For instance the first note played here is distorted0:note:s5:f0 (string 5, fret 0).
It is followed by a nfx token "nfx:palm\_mute", which is a note effect token that applies to the previous note.
Finally, the duration of the note is given by the wait token "wait:480" (480 ticks in GuitarPro correspond to an eighth note).
If no tokens of a specific instrument are present between two wait tokens, it means that the instrument keeps on playing the same note.
This is the case of the bass in the example.
Its first quarter note lasts for 960 ticks, therefore we have to look after the three first wait tokens (480 + 240 + 240) before seeing a new token for the bass.


% Compound Gen tokenization
As we plan to use the model built by Makris et al. in 2022, we will also consider the tokenization they used.
The proposed tokenization model introduces a compound word (CP) representation, inspired by previous CP-based approaches\cite{hsiao_compound_2021}.
Unlike traditional tokenization strategies such as MIDI-like or REMI, which encode music as a linear sequence of tokens, CP groups multiple features of a musical event into a single "super token" or "word".
This n-dimensional representation significantly reduces sequence length and improves computational efficiency in Transformer-based architectures\cite{makris_conditional_2022}.


In this model, the CP representation is adapted for an Encoder-Decoder architecture to condition drum generation on accompanying rhythm tracks, such as bass and guitar.
The Encoder processes a 5-dimensional CP representation that includes rhythm track features and high-level musical information like tempo and time signature, which are known to influence the complexity and density of drum tracks.
Both the Encoder and Decoder adopt bar-based segmentation, consistent with existing CP and REMI representations.
By combining rhythm section inputs and high-level parameters, this tokenization approach aligns with the fundamental musical principles of contemporary western music's rhythm section\cite{makris_conditional_2022}.
As said previously, our work will use the DadaGP tokenization. However, the nature of the generation task we want to perform is very similar to the one of Makris et al. in the sense that bass guitar and drums are part of the same rhythm section.


\begin{figure}[h!]
    \centering
    \includegraphics[width=.5\linewidth]{../images-figures/tokenization_makris.png}
    \caption{Example of a CP representation used for training. Figure taken from \cite{makris_conditional_2022}}
    \label{fig:compound_gen_tokenization}
\end{figure}


Figure \ref{fig:compound_gen_tokenization} shows an example of a CP representation used for training in the model developed by Makris et al.
We notice that the rhythmic section, composed of a rhythmic guitar and a bass (in blue on the figure), constitutes the conditioning part of the model.
Their representation as tokens encode the duration and type of the notes played, but not the pitch of the notes as it is not relevant for the drum generation task.
In grey, the drum part's encoding excludes all information about velocities and duration.
Drums notes and are only represented by their onset in the bar and the drum component played.

\subsection{Conditional generation models}

% Intro about conditional generation using control tokens (prompt tokens)
% --> GTR CTRL, ShredGP, ProgGP
Our first idea was to use the GuitarCTRL model developed by Sarmento et al. in 2021\cite{sarmento_gtr-ctrl_2023}.
The GTR CTRL model utilizes a Transformer-XL architecture\cite{dai_transformer-xl_2019}, which improves on the vanilla Transformer by introducing recurrence and modified positional encodings, enabling long-term dependency learning.
This model includes two key control types: instrumentation (inst-CTRL) and genre (genre-CTRL).
For instrumentation, tokens marking the start and end of each instrument are inserted into the header, guiding the model to generate music for specified instruments, such as bass and drums or distorted guitar and drums.
Genre control tokens are similarly placed at the beginning of the sequence, with a dataset covering over 200 songs per genre, including Metal, Rock, Punk, Folk, and Classical.
The model uses various prompts at inference, ranging from full-prompt (including two measures plus the genre token) to empty-prompt (only the genre token).
This model was used to generate a baseline result for our project, conditioning the generation to bass guitar only.


% Start with a historic of BI-LSTM in NLP (2015)
% Dive in makris et al model (BiLSTM) (SOTA in drum generation)
The model we wish to use for our project is the one developed by Makris et al. in 2022\cite{makris_conditional_2022}.
It employs an Encoder-Decoder architecture with a BiLSTM Encoder and a Transformer Decoder utilizing Relative Global Attention.
The Encoder is made of several BiLSTM layers. It processes Compound Representation (CP) inputs to generate a latent variable z that encodes high-level features of the input and conditional tracks.
BiLSTM (Bidirectional Long Short-Term Memory) were first introduced by Graves et al. in 2005\cite{graves_framewise_2005}. It is a type of recurrent neural network (RNN) that processes data in both forward and backward directions.
This allows the model to consider both past (previous time steps) and future (upcoming time steps) context in the sequence
The Decoder consists of self-attention layers and feed-forward layers. 
It combines z with previous timestep embeddings to predict drum onsets and pitches.



The model uses input sizes that are different from the ones we may use with our tokenization.
However we are interested in the ability of the model to generate conditionnally throughout the whole sequence.
In comparison, GTR CTRL only conditions the generation at the beginning of the sequence.
One of the issue we encountered when generating bass guitar was that other instruments ended up being generated further in the sequence.
To avoid this we tried setting the probability of predicting tokens from any other instruments to 0, but even then the bass lines generated were not coherent.
This model is simply not suited for our task which explains our motivations to adapt the BiLSTM model.


\section{Data preprocessing}


To prepare the data for training, we follow a structured preprocessing pipeline that ensures clean and well-formatted input sequences.
The first step involves retrieving and cleaning the DadaGP dataset, extracting tokens specific to selected instruments while preserving relevant note effect tokens and metadata.
We then apply an existing method to detect rhythm guitar parts within the dataset, selecting the most rhythmic guitar track for conditioning bass guitar generation.  
Next, we segment the extracted tokens into overlapping sequences of 16 measures with a stride of 8, applying length constraints to remove excessively long or short sequences.
The selected sequences are then formatted into structured dictionaries, where rhythm guitar tokens are used as encoder inputs, and bass guitar tokens serve as decoder targets.  
Finally, the sequences are converted to token indices, padded to fixed lengths, and split into training, validation, and test sets for model development and evaluation.
The following subsections provide a detailed explanation of each of these steps.  

\subsection{DadaGP token preprocessing}
% --> Retrieving, cleaning, mapping DadaGP
% Rhythm guitar detection (cite paper)

The DadaGP dataset provides token files containing information for all instruments in a score.
As we want to explore various conditioning combinations for tablature generation, we developed a function to extract tokens specific to selected instruments.
We considered several possible ways to perform this extraction: runnning DadaGP's token processing script for each instrument, or filtering the tokens from the complete token text file based on the instrument name.
Opening, parsing and looping on the GuitarPro file using \texttt{pyGuitarPro}\footnote{https://pyguitarpro.readthedocs.io/en/stable/} is needed for the DadaGP token processing script and is very time-consuming, much more than processing text files.
To perform the extraction of instrument-specific tokens, we use the filtering method from the existing token text files.
We leverage the fact that tokens of a given instrument start with the instrument name followed by a colon, e.g., \texttt{bass:} for the bass guitar.
However we must be careful not to miss the potential note effect tokens that are not instrument-specific, but come right after the token they are related to.
After extracting those tokens and the general tokens (metadata tokens and wait tokens),
we sum the potential consecutive wait tokens that were previously separated by notes from instruments that are no longer present.
This process took 5min30 on the 26,181 GuitarPro files of DadaGP to generate a copy of the dataset containing token files of bass guitar only.

\begin{figure}[!ht]
    \centering
    \includegraphics[width=.4\linewidth]{../images-figures/token_extraction.png}
    \caption{Example of an extraction of bass guitar tokens in The Strokes - Last Nite}
    \label{fig:token_extraction}
\end{figure}

An example of token extraction is shown in Figure~\ref{fig:token_extraction}.
Note from \texttt{distorted0}, \texttt{distorted1} and \texttt{drums} are removed as well as \texttt{bfx} tokens.
This figure also shows the concatenation of wait tokens, at the end of the song the bass guitar stops playing before the other instruments.
Therefore, we can notice that eight \texttt{wait:480} tokens (some are not shown here) are summed up to a single \texttt{wait:3840} token.



\subsection{Rhythm guitar identification}

The algorithm we just presented allows us to extract tokens for any instrument in the DadaGP dataset.
In typical rock band instrumentation, rhythm guitars complement bass guitars with their repetitive chord patterns, providing harmonic and rhythmic structure at a higher pitch~\cite{regnier_identification_2021}.
This characteristic makes them particularly relevant for conditioning bass generation tasks. Therefore, our first focus is on conditioning bass guitar tablature generation using the rhythm guitar.
However, as part of the rhythm section, drums and bass also form a crucial pair in the band.
Conditional generation using the drums part was considered, but the absence of pitch and harmonic information in drums made it less relevant for our task.


To identify rhythm guitar parts in the DadaGP dataset, we implemented the method proposed by Régnier et al.~\cite{regnier_identification_2021},
which uses features describing notes and chords at the bar level, along with corresponding tablatures.
Their model outputs a prediction between 0 and 1 for each bar. The closest to 1, the more likely the bar is to be lead guitar, and the closest to 0, the more likely the bar is to be rhythm guitar.
In their paper, they consider that a score below 0.5 correspond to a rhythm guitar parts.
As instrumentation can greatly vary between songs and genres, we wanted to select and condition the generation on only one guitar part, the ``most" rhythmic among the guitar parts present.
Therefore, we applied the 0.5 threshold to the predictions and selected the part with the highest proportion of rhythmic bar over the course of the song.
Implementing this method required adapting their code to the DadaGP dataset.
Significant effort was spent mapping the identified rhythm guitar parts to their respective names in the DadaGP files, as the dataset renames instruments, whereas the identification tool relies on GuitarPro part names.
Thanks to this work, we were able to extra  ct two more datasets: one containing the rhythm guitar tokens and the other containing both the bass and the rhythm guitar tokens.

% FIGURE EXAMPLE OF SELECTION OF RHYTHM GUITAR ON A SONG
\begin{figure}[!ht]
    \centering
    \includegraphics[width=0.85\linewidth]{../images-figures/rhythm_guitar_selection.png}
    \caption{Example of rhythm guitar selection after applying the method of Régnier et al.}
    \label{fig:rhythm_guitar_selection}
\end{figure}

Figure~\ref{fig:rhythm_guitar_selection} shows an example of the selection of the rhythm guitar part.
After retrieving the instrument with the highest number of rhythmic bars over the song,
we generate a column \texttt{is\_track\_rhythmic} that is filled with 1 (or True) for the rhythm guitar and 0 for the other instruments.


\subsection{Post processing}

To avoid sequences that are too long, we chose to extract samples of 16 measures from the songs.
We extract sequences with a stride of 8 measures, which allows us to have a good overlap between the sequences.
This will help the model to learn the transitions, and understand the various possible contexts around a given sequence.

% FIGURE DISTRIBUTION OF SEQUENCE LENGTHS
\begin{figure}[!ht]
    \centering
    \includegraphics[width=.8\linewidth]{../images-figures/sequence_lengths_16_8_800_50.png}
    \caption{Distribution of sequence lengths in the training dataset}
    \label{fig:sequence_length_distribution}
\end{figure}

A maximum threshold of 800 tokens, and a minimum threshold of 50 tokens were set to avoid training on sequences that are either too long or that does not contain rythmic guitar or bass.
These are not applied on the complete sequences containing both the tokens of the rhythm guitar and the bass.
Figure~\ref{fig:sequence_length_distribution} shows the distribution of sequence lengths in the training dataset.
The majority of the sequences are between 200 and 400 tokens long for the rythmic guitar and between 100 and 300 tokens long for the bass.
Then, the sequences are fitted to the model's input requirements using an adapted version of the processing process of Makris et al.~\cite{makris_conditional_2022}.
The sequences fill in a dictionary with keys \texttt{Encoder\_RG}, \texttt{Decoder\_Bass} and \texttt{All\_Events}.
For instance \texttt{dict}[\texttt{Encoder\_RG}][i] contains the tokens of the rhythm guitar for the i-th sequence.
\texttt{dict}[\texttt{All\_Events}][i] contains the tokens of bass guitar and rhythm guitar for the i-th sequence.


\begin{figure}[!ht]
    \centering
    \includegraphics[width=.8\linewidth]{../images-figures/conversion_indices.png}
    \caption{Conversion from a text token sequence to indices}
    \label{fig:conversion_indices}
\end{figure}

The sequences are then processed as described in Figure~\ref{fig:conversion_indices}.
They go through a one-hot encoding process, then we add 'start of sequence' and 'end of sequence',
and padding tokens after the 'eos' token to reach a fixed length of 797 tokens for the encoder (Rhythm guitar) and 773 tokens for the decoder (Bass guitar).
The final dictionary of sequences as indices is finally split into training, validation and test sets with proportions 0.8, 0.1 and 0.1 respectively.
We generated 118 167 sequences in total, which leads to 94 533 sequences in the training set and 11 817 in both the validation and test sets.
Those sets are respectively used to train the model, tune the hyperparameters and evaluate the model's performance.



\section{Models}

\subsection{Baseline predictions}

We began by establishing a baseline model for bass guitar tablature generation.
Initially, we utilized a pre-trained checkpoint from the DadaGP paper \cite{sarmento_dadagp_2021} without additional training.

To generate bass-specific outputs, we experimented with various prompts.
First, we provided a single bass guitar note as the initial token, however the model quickly incorporated tokens from other instruments.
This is an issue documented in the GuitarCTRL paper\cite{sarmento_gtr-ctrl_2023} where they used a metric called the UIP score (Unpromped Instrument Presence) to evaluate the model's ability to generate a specific combinations of instruments.
To constrain the output to bass tokens, we modified the logits of non-bass instruments by setting them to $-\infty$.

% FIGURE REPETITIVE GENERATION
\begin{figure}[!ht]
    \centering
    \includegraphics[width=.75\linewidth]{../images-figures/generated_bass_baseline.png}
    \caption{Example of a bass generation from the baseline model}
    \label{fig:repetitive_generation}
\end{figure}

An example of a bass tablature generated this way is shown in Figure~\ref{fig:repetitive_generation}.
While this forced the model to generate only bass tokens, the output quality was poor, featuring repetitive phrases, harmonically incorrect notes, or complete aberrations.
This was expected, as the model was not trained explicitly for bass only generation. 
Unfortunately, we cannot provide a quantitative evaluation of the generated tablatures, as we have not yet defined an evaluation metric.

\subsection{BiLSTM - Transformer model}

This section present the results we get using a model adapted from the work of Makris et al. \cite{makris_conditional_2022}.
Our data was adapted to fit the model's input requirements and the model was also slightly modified to fit our task.

% FIGURE MAKRIS ET AL. MODEL
\begin{figure}[!ht]
    \centering
    \includegraphics[width=.75\linewidth]{../images-figures/makris_model.png}
    \caption{Makris et al. model architecture, taken from their paper \cite{makris_conditional_2022}}
    \label{fig:makris_model}
\end{figure}

The figure~\ref{fig:makris_model} shows the architecture of the model proposed by Makris et al.
We have presented the compound representation in the state of the art.
This representation concatenates all the parameters necessary to describe a note in a single vector (onset, duration, pitch etc.).

% FIGURE MAKRIS ET AL. MODEL ADAPTED TO OUR TASK
\begin{figure}[!ht]
    \centering
    \includegraphics[width=.75\linewidth]{../images-figures/model_adapted.png}
    \caption{Makris et al. model adapted to our task}
    \label{fig:makris_model_adapted}
\end{figure}

Figure \ref{fig:makris_model_adapted} shows the model adapted to our task.
In our case, we have chosen to use DadaGP tokenization which is a different representation.
On the encoder side, the five embeddings concatenated in Makris et al. are replaced by a single embedding in our case.
On the decoder side, their input had a special shape because drums notes need much less parameters than other instruments.
They are fully described by their onset and a value indicating the drum type.
This compound representation is also replaced by a single embedding in our case.

The embeddings are then passed through a dense layer. The encoder input (rythmic guitar) is passed through a BiLSTM layer.
Positional encodings are added to the decoder input and the output of the BiLSTM layer.
All these elements are then passed through a transformer layer with self-attention mechanism
that outputs an array of vocabulary size with the probability of each token.

\subsection{Training and inference}

The model was trained on 11 020 extract from songs in the DadaGP dataset.
These extracts are the first 16 measures of the song, and contain sequences of at most 800 tokens and at least 50 tokens.
Thresholds were set to avoid training on sequences that are either too long or that does not contain rythmic guitar or bass.

% FIGURE DISTRIBUTION OF SEQUENCE LENGTHS
\begin{figure}[!ht]
    \centering
    \includegraphics[width=.75\linewidth]{../images-figures/sequence_lengths_16_800_50.png}
    \caption{Distribution of sequence lengths in the training dataset}
    \label{fig:sequence_length_distribution}
\end{figure}

Figure \ref{fig:sequence_length_distribution} shows the distribution of sequence lengths in the training dataset.
The majority of the sequences are between 200 and 400 tokens long for the rythmic guitar and between 100 and 300 tokens long for the bass.

Training was done on GPU with 200 epochs and a batch size of 32.
However the training stopped at epoch 29 because the validation loss did not decrease anymore for 5 epochs.
We reached 0.90 accuracy on the validation set with 0.42 loss.

\section{Results, discussion and perspectives}

\subsection{Results}

We did not have time to implement quantitative evaluation metrics beyond accuracy and loss, so we rely on our expertise to assess the quality of the generated tablatures.
Inference was conducted on the entire test set of 11,817 sequences.
The model achieved an accuracy of [VALUE TO FILL] on those sequences, with a corresponding loss of [VALUE TO FILL].
These values are very close to those observed on the validation set.

In this section, we showcase a selection of results from our experiments.
The examples were carefully chosen to demonstrate the model's ability to generate tablatures across different musical genres and diverse rhythmic guitar patterns.
We manually selected rhythmic guitar parts and extracted their corresponding tokens.
These examples originate from Songsterr \footnote{https://www.songsterr.com/} but are not part of the training dataset.
However, they belong to genres that are well-represented in DadaGP: various types of rock, metal and jazz funk.
Our work is the first to generate bass tablatures conditioned on rhythmic guitar parts, so we do not have other models to compare our results to.

\subsection{Examples}

% FIGURE EXAMPLE 1: METAL: ULTRA VOMIT RICARD PEINARD

\begin{figure}[!ht]
    \centering
    \includegraphics[width=\linewidth]{../images-figures/gen_ricard_peinard.png}
    \caption{Example of 9 measures generated by the model based on the rhythmic guitar part of the song "Ricard Peinard" by Ultra Vomit}
    \label{fig:gen_ricard_peinard}
\end{figure}

Figure \ref{fig:gen_ricard_peinard} shows an excerpt of bass tablatures generated by the model in the metal genre.
Bass guitar in metal music often plays a supportive role, doubling the rhythm guitar part at a lower pitch.
We also display the rhythmic guitar part and the original bass part.
Several criterion can be qualitatively used to evaluate the model's performance.
First, we can notice that the generated part is harmonically coherent with the rhythmic guitar part:
on the first measure the bass plays several F notes, while the rhythmic guitar plays a chord composed of F, C and G notes.
Rhythmically, the generated part either copies the rhythmic guitar or plays a simpler pattern such as a succession of eighth notes.
This is a correct outcome in the context of metal music, but needs to be further evaluated in other genres.
Finally, one may notice that the generated part is identical to the original bass part on measures 5 to 8.


% FIGURE EXAMPLE 2: INDIE ROCK: ARCTIC MONKEYS DO I WANNA KNOW

\begin{figure}[!ht]
    \centering
    \includegraphics[width=\linewidth]{../images-figures/gen_arctic_monkeys.png}
    \caption{Example of 9 measures generated by the model based on the rhythmic guitar part of the song "Do I Wanna Know" by Arctic Monkeys}
    \label{fig:gen_arctic_monkeys}
\end{figure}

Figure \ref{fig:gen_arctic_monkeys} shows an excerpt of bass tablatures generated by the model in the indie rock genre.
Bass guitar's role in indie rock music is hard to define due to the genre's diversity.
However, this example remains particularly interesting as it demonstrates the model's tendency to copy the rhythmic guitar part even when very simple parts would be more appropriate.
Here the original bass part is basic yet essential, providing groove and structure to the song using quarter notes at specific onsets.
Nonetheless, the generated part remains harmonically and rhythmically coherent with the rhythmic guitar part.
This example also shows a recurrent issue in the model's inability to follow a time signature.
Here measure 12 becomes a 7/8 instead of a 4/4 (the model forgot an eighth note in the measure).


% FIGURE EXAMPLE 3: JAZZ FUNK: HERBIE HANCOCK SATURDAY NIGHT

\begin{figure}[!ht]
    \centering
    \includegraphics[width=\linewidth]{../images-figures/gen_herbie_hancock.png}
    \caption{Example of 9 measures generated by the model based on the rhythmic guitar part of the song "Saturday Night" by Herbie Hancock}
    \label{fig:gen_herbie_hancock}
\end{figure}

Figure \ref{fig:gen_herbie_hancock} shows an excerpt of bass tablatures generated by the model in the jazz funk genre.
Bass guitar in jazz funk music often plays a more melodic role, with complex rhythmic patterns and harmonies.
Moreover, there are no rhythmic guitar parts strictly speaking in jazz funk music, so the model generates using a more melodic guitar part.
As expected, the generated part is way simpler than the original bass part.
It is meant to be an accompaniment to the guitar part, not a melodic part.
Moreover, jazz funk music often uses complex harmonies, so the model has a hard time following the harmonic progression.
This genre can be considered as a limit case for the model. As it was trained on an agglomerate of genres, it is not able to generate coherent jazz funk bass lines.

To summarize, the model has great potential in generating bass tablatures of accompaniment that are harmonically and rhythmically coherent with the rhythmic guitar part.
However, it struggles to adapt to the specificities of each genre and can still make rhythmical mistakes.
We tried to build an adaptable model that could generate bass lines for a wide range of styles,
but it would be interesting to analyze the model's performance on a specific genre if trained on said genre only.

\subsection{Perspectives}

With additional time, several improvements could be made to the model architecture and tuning, data processing, and evaluation.
One potential enhancement is simplifying the model by removing one of the dense layers in the embedding process, which could reduce complexity without significantly impacting performance.
Additionally, experimenting with compound word tokenization as an alternative to the current DADAGP method might yield more meaningful representations. 

Beyond architectural changes, a user study could be conducted to assess the quality of the generated bass lines, providing valuable insights into the model's musical coherence.
Finally, evaluating the model's environmental impact using Green AI metrics, specifically by measuring the number of floating point operations (FLOPs), would contribute to a better understanding of its computational efficiency.



\newpage

\bibliographystyle{apalike}   
\bibliography{references}  

\end{document}